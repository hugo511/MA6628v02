
% Default to the notebook output style

    


% Inherit from the specified cell style.




    
\documentclass[11pt]{article}

    
    
    \usepackage[T1]{fontenc}
    % Nicer default font (+ math font) than Computer Modern for most use cases
    \usepackage{mathpazo}

    % Basic figure setup, for now with no caption control since it's done
    % automatically by Pandoc (which extracts ![](path) syntax from Markdown).
    \usepackage{graphicx}
    % We will generate all images so they have a width \maxwidth. This means
    % that they will get their normal width if they fit onto the page, but
    % are scaled down if they would overflow the margins.
    \makeatletter
    \def\maxwidth{\ifdim\Gin@nat@width>\linewidth\linewidth
    \else\Gin@nat@width\fi}
    \makeatother
    \let\Oldincludegraphics\includegraphics
    % Set max figure width to be 80% of text width, for now hardcoded.
    \renewcommand{\includegraphics}[1]{\Oldincludegraphics[width=.8\maxwidth]{#1}}
    % Ensure that by default, figures have no caption (until we provide a
    % proper Figure object with a Caption API and a way to capture that
    % in the conversion process - todo).
    \usepackage{caption}
    \DeclareCaptionLabelFormat{nolabel}{}
    \captionsetup{labelformat=nolabel}

    \usepackage{adjustbox} % Used to constrain images to a maximum size 
    \usepackage{xcolor} % Allow colors to be defined
    \usepackage{enumerate} % Needed for markdown enumerations to work
    \usepackage{geometry} % Used to adjust the document margins
    \usepackage{amsmath} % Equations
    \usepackage{amssymb} % Equations
    \usepackage{textcomp} % defines textquotesingle
    % Hack from http://tex.stackexchange.com/a/47451/13684:
    \AtBeginDocument{%
        \def\PYZsq{\textquotesingle}% Upright quotes in Pygmentized code
    }
    \usepackage{upquote} % Upright quotes for verbatim code
    \usepackage{eurosym} % defines \euro
    \usepackage[mathletters]{ucs} % Extended unicode (utf-8) support
    \usepackage[utf8x]{inputenc} % Allow utf-8 characters in the tex document
    \usepackage{fancyvrb} % verbatim replacement that allows latex
    \usepackage{grffile} % extends the file name processing of package graphics 
                         % to support a larger range 
    % The hyperref package gives us a pdf with properly built
    % internal navigation ('pdf bookmarks' for the table of contents,
    % internal cross-reference links, web links for URLs, etc.)
    \usepackage{hyperref}
    \usepackage{longtable} % longtable support required by pandoc >1.10
    \usepackage{booktabs}  % table support for pandoc > 1.12.2
    \usepackage[inline]{enumitem} % IRkernel/repr support (it uses the enumerate* environment)
    \usepackage[normalem]{ulem} % ulem is needed to support strikethroughs (\sout)
                                % normalem makes italics be italics, not underlines
    

    
    
    % Colors for the hyperref package
    \definecolor{urlcolor}{rgb}{0,.145,.698}
    \definecolor{linkcolor}{rgb}{.71,0.21,0.01}
    \definecolor{citecolor}{rgb}{.12,.54,.11}

    % ANSI colors
    \definecolor{ansi-black}{HTML}{3E424D}
    \definecolor{ansi-black-intense}{HTML}{282C36}
    \definecolor{ansi-red}{HTML}{E75C58}
    \definecolor{ansi-red-intense}{HTML}{B22B31}
    \definecolor{ansi-green}{HTML}{00A250}
    \definecolor{ansi-green-intense}{HTML}{007427}
    \definecolor{ansi-yellow}{HTML}{DDB62B}
    \definecolor{ansi-yellow-intense}{HTML}{B27D12}
    \definecolor{ansi-blue}{HTML}{208FFB}
    \definecolor{ansi-blue-intense}{HTML}{0065CA}
    \definecolor{ansi-magenta}{HTML}{D160C4}
    \definecolor{ansi-magenta-intense}{HTML}{A03196}
    \definecolor{ansi-cyan}{HTML}{60C6C8}
    \definecolor{ansi-cyan-intense}{HTML}{258F8F}
    \definecolor{ansi-white}{HTML}{C5C1B4}
    \definecolor{ansi-white-intense}{HTML}{A1A6B2}

    % commands and environments needed by pandoc snippets
    % extracted from the output of `pandoc -s`
    \providecommand{\tightlist}{%
      \setlength{\itemsep}{0pt}\setlength{\parskip}{0pt}}
    \DefineVerbatimEnvironment{Highlighting}{Verbatim}{commandchars=\\\{\}}
    % Add ',fontsize=\small' for more characters per line
    \newenvironment{Shaded}{}{}
    \newcommand{\KeywordTok}[1]{\textcolor[rgb]{0.00,0.44,0.13}{\textbf{{#1}}}}
    \newcommand{\DataTypeTok}[1]{\textcolor[rgb]{0.56,0.13,0.00}{{#1}}}
    \newcommand{\DecValTok}[1]{\textcolor[rgb]{0.25,0.63,0.44}{{#1}}}
    \newcommand{\BaseNTok}[1]{\textcolor[rgb]{0.25,0.63,0.44}{{#1}}}
    \newcommand{\FloatTok}[1]{\textcolor[rgb]{0.25,0.63,0.44}{{#1}}}
    \newcommand{\CharTok}[1]{\textcolor[rgb]{0.25,0.44,0.63}{{#1}}}
    \newcommand{\StringTok}[1]{\textcolor[rgb]{0.25,0.44,0.63}{{#1}}}
    \newcommand{\CommentTok}[1]{\textcolor[rgb]{0.38,0.63,0.69}{\textit{{#1}}}}
    \newcommand{\OtherTok}[1]{\textcolor[rgb]{0.00,0.44,0.13}{{#1}}}
    \newcommand{\AlertTok}[1]{\textcolor[rgb]{1.00,0.00,0.00}{\textbf{{#1}}}}
    \newcommand{\FunctionTok}[1]{\textcolor[rgb]{0.02,0.16,0.49}{{#1}}}
    \newcommand{\RegionMarkerTok}[1]{{#1}}
    \newcommand{\ErrorTok}[1]{\textcolor[rgb]{1.00,0.00,0.00}{\textbf{{#1}}}}
    \newcommand{\NormalTok}[1]{{#1}}
    
    % Additional commands for more recent versions of Pandoc
    \newcommand{\ConstantTok}[1]{\textcolor[rgb]{0.53,0.00,0.00}{{#1}}}
    \newcommand{\SpecialCharTok}[1]{\textcolor[rgb]{0.25,0.44,0.63}{{#1}}}
    \newcommand{\VerbatimStringTok}[1]{\textcolor[rgb]{0.25,0.44,0.63}{{#1}}}
    \newcommand{\SpecialStringTok}[1]{\textcolor[rgb]{0.73,0.40,0.53}{{#1}}}
    \newcommand{\ImportTok}[1]{{#1}}
    \newcommand{\DocumentationTok}[1]{\textcolor[rgb]{0.73,0.13,0.13}{\textit{{#1}}}}
    \newcommand{\AnnotationTok}[1]{\textcolor[rgb]{0.38,0.63,0.69}{\textbf{\textit{{#1}}}}}
    \newcommand{\CommentVarTok}[1]{\textcolor[rgb]{0.38,0.63,0.69}{\textbf{\textit{{#1}}}}}
    \newcommand{\VariableTok}[1]{\textcolor[rgb]{0.10,0.09,0.49}{{#1}}}
    \newcommand{\ControlFlowTok}[1]{\textcolor[rgb]{0.00,0.44,0.13}{\textbf{{#1}}}}
    \newcommand{\OperatorTok}[1]{\textcolor[rgb]{0.40,0.40,0.40}{{#1}}}
    \newcommand{\BuiltInTok}[1]{{#1}}
    \newcommand{\ExtensionTok}[1]{{#1}}
    \newcommand{\PreprocessorTok}[1]{\textcolor[rgb]{0.74,0.48,0.00}{{#1}}}
    \newcommand{\AttributeTok}[1]{\textcolor[rgb]{0.49,0.56,0.16}{{#1}}}
    \newcommand{\InformationTok}[1]{\textcolor[rgb]{0.38,0.63,0.69}{\textbf{\textit{{#1}}}}}
    \newcommand{\WarningTok}[1]{\textcolor[rgb]{0.38,0.63,0.69}{\textbf{\textit{{#1}}}}}
    
    
    % Define a nice break command that doesn't care if a line doesn't already
    % exist.
    \def\br{\hspace*{\fill} \\* }
    % Math Jax compatability definitions
    \def\gt{>}
    \def\lt{<}
    % Document parameters
    \title{L02s01}
    
    
    

    % Pygments definitions
    
\makeatletter
\def\PY@reset{\let\PY@it=\relax \let\PY@bf=\relax%
    \let\PY@ul=\relax \let\PY@tc=\relax%
    \let\PY@bc=\relax \let\PY@ff=\relax}
\def\PY@tok#1{\csname PY@tok@#1\endcsname}
\def\PY@toks#1+{\ifx\relax#1\empty\else%
    \PY@tok{#1}\expandafter\PY@toks\fi}
\def\PY@do#1{\PY@bc{\PY@tc{\PY@ul{%
    \PY@it{\PY@bf{\PY@ff{#1}}}}}}}
\def\PY#1#2{\PY@reset\PY@toks#1+\relax+\PY@do{#2}}

\expandafter\def\csname PY@tok@w\endcsname{\def\PY@tc##1{\textcolor[rgb]{0.73,0.73,0.73}{##1}}}
\expandafter\def\csname PY@tok@c\endcsname{\let\PY@it=\textit\def\PY@tc##1{\textcolor[rgb]{0.25,0.50,0.50}{##1}}}
\expandafter\def\csname PY@tok@cp\endcsname{\def\PY@tc##1{\textcolor[rgb]{0.74,0.48,0.00}{##1}}}
\expandafter\def\csname PY@tok@k\endcsname{\let\PY@bf=\textbf\def\PY@tc##1{\textcolor[rgb]{0.00,0.50,0.00}{##1}}}
\expandafter\def\csname PY@tok@kp\endcsname{\def\PY@tc##1{\textcolor[rgb]{0.00,0.50,0.00}{##1}}}
\expandafter\def\csname PY@tok@kt\endcsname{\def\PY@tc##1{\textcolor[rgb]{0.69,0.00,0.25}{##1}}}
\expandafter\def\csname PY@tok@o\endcsname{\def\PY@tc##1{\textcolor[rgb]{0.40,0.40,0.40}{##1}}}
\expandafter\def\csname PY@tok@ow\endcsname{\let\PY@bf=\textbf\def\PY@tc##1{\textcolor[rgb]{0.67,0.13,1.00}{##1}}}
\expandafter\def\csname PY@tok@nb\endcsname{\def\PY@tc##1{\textcolor[rgb]{0.00,0.50,0.00}{##1}}}
\expandafter\def\csname PY@tok@nf\endcsname{\def\PY@tc##1{\textcolor[rgb]{0.00,0.00,1.00}{##1}}}
\expandafter\def\csname PY@tok@nc\endcsname{\let\PY@bf=\textbf\def\PY@tc##1{\textcolor[rgb]{0.00,0.00,1.00}{##1}}}
\expandafter\def\csname PY@tok@nn\endcsname{\let\PY@bf=\textbf\def\PY@tc##1{\textcolor[rgb]{0.00,0.00,1.00}{##1}}}
\expandafter\def\csname PY@tok@ne\endcsname{\let\PY@bf=\textbf\def\PY@tc##1{\textcolor[rgb]{0.82,0.25,0.23}{##1}}}
\expandafter\def\csname PY@tok@nv\endcsname{\def\PY@tc##1{\textcolor[rgb]{0.10,0.09,0.49}{##1}}}
\expandafter\def\csname PY@tok@no\endcsname{\def\PY@tc##1{\textcolor[rgb]{0.53,0.00,0.00}{##1}}}
\expandafter\def\csname PY@tok@nl\endcsname{\def\PY@tc##1{\textcolor[rgb]{0.63,0.63,0.00}{##1}}}
\expandafter\def\csname PY@tok@ni\endcsname{\let\PY@bf=\textbf\def\PY@tc##1{\textcolor[rgb]{0.60,0.60,0.60}{##1}}}
\expandafter\def\csname PY@tok@na\endcsname{\def\PY@tc##1{\textcolor[rgb]{0.49,0.56,0.16}{##1}}}
\expandafter\def\csname PY@tok@nt\endcsname{\let\PY@bf=\textbf\def\PY@tc##1{\textcolor[rgb]{0.00,0.50,0.00}{##1}}}
\expandafter\def\csname PY@tok@nd\endcsname{\def\PY@tc##1{\textcolor[rgb]{0.67,0.13,1.00}{##1}}}
\expandafter\def\csname PY@tok@s\endcsname{\def\PY@tc##1{\textcolor[rgb]{0.73,0.13,0.13}{##1}}}
\expandafter\def\csname PY@tok@sd\endcsname{\let\PY@it=\textit\def\PY@tc##1{\textcolor[rgb]{0.73,0.13,0.13}{##1}}}
\expandafter\def\csname PY@tok@si\endcsname{\let\PY@bf=\textbf\def\PY@tc##1{\textcolor[rgb]{0.73,0.40,0.53}{##1}}}
\expandafter\def\csname PY@tok@se\endcsname{\let\PY@bf=\textbf\def\PY@tc##1{\textcolor[rgb]{0.73,0.40,0.13}{##1}}}
\expandafter\def\csname PY@tok@sr\endcsname{\def\PY@tc##1{\textcolor[rgb]{0.73,0.40,0.53}{##1}}}
\expandafter\def\csname PY@tok@ss\endcsname{\def\PY@tc##1{\textcolor[rgb]{0.10,0.09,0.49}{##1}}}
\expandafter\def\csname PY@tok@sx\endcsname{\def\PY@tc##1{\textcolor[rgb]{0.00,0.50,0.00}{##1}}}
\expandafter\def\csname PY@tok@m\endcsname{\def\PY@tc##1{\textcolor[rgb]{0.40,0.40,0.40}{##1}}}
\expandafter\def\csname PY@tok@gh\endcsname{\let\PY@bf=\textbf\def\PY@tc##1{\textcolor[rgb]{0.00,0.00,0.50}{##1}}}
\expandafter\def\csname PY@tok@gu\endcsname{\let\PY@bf=\textbf\def\PY@tc##1{\textcolor[rgb]{0.50,0.00,0.50}{##1}}}
\expandafter\def\csname PY@tok@gd\endcsname{\def\PY@tc##1{\textcolor[rgb]{0.63,0.00,0.00}{##1}}}
\expandafter\def\csname PY@tok@gi\endcsname{\def\PY@tc##1{\textcolor[rgb]{0.00,0.63,0.00}{##1}}}
\expandafter\def\csname PY@tok@gr\endcsname{\def\PY@tc##1{\textcolor[rgb]{1.00,0.00,0.00}{##1}}}
\expandafter\def\csname PY@tok@ge\endcsname{\let\PY@it=\textit}
\expandafter\def\csname PY@tok@gs\endcsname{\let\PY@bf=\textbf}
\expandafter\def\csname PY@tok@gp\endcsname{\let\PY@bf=\textbf\def\PY@tc##1{\textcolor[rgb]{0.00,0.00,0.50}{##1}}}
\expandafter\def\csname PY@tok@go\endcsname{\def\PY@tc##1{\textcolor[rgb]{0.53,0.53,0.53}{##1}}}
\expandafter\def\csname PY@tok@gt\endcsname{\def\PY@tc##1{\textcolor[rgb]{0.00,0.27,0.87}{##1}}}
\expandafter\def\csname PY@tok@err\endcsname{\def\PY@bc##1{\setlength{\fboxsep}{0pt}\fcolorbox[rgb]{1.00,0.00,0.00}{1,1,1}{\strut ##1}}}
\expandafter\def\csname PY@tok@kc\endcsname{\let\PY@bf=\textbf\def\PY@tc##1{\textcolor[rgb]{0.00,0.50,0.00}{##1}}}
\expandafter\def\csname PY@tok@kd\endcsname{\let\PY@bf=\textbf\def\PY@tc##1{\textcolor[rgb]{0.00,0.50,0.00}{##1}}}
\expandafter\def\csname PY@tok@kn\endcsname{\let\PY@bf=\textbf\def\PY@tc##1{\textcolor[rgb]{0.00,0.50,0.00}{##1}}}
\expandafter\def\csname PY@tok@kr\endcsname{\let\PY@bf=\textbf\def\PY@tc##1{\textcolor[rgb]{0.00,0.50,0.00}{##1}}}
\expandafter\def\csname PY@tok@bp\endcsname{\def\PY@tc##1{\textcolor[rgb]{0.00,0.50,0.00}{##1}}}
\expandafter\def\csname PY@tok@fm\endcsname{\def\PY@tc##1{\textcolor[rgb]{0.00,0.00,1.00}{##1}}}
\expandafter\def\csname PY@tok@vc\endcsname{\def\PY@tc##1{\textcolor[rgb]{0.10,0.09,0.49}{##1}}}
\expandafter\def\csname PY@tok@vg\endcsname{\def\PY@tc##1{\textcolor[rgb]{0.10,0.09,0.49}{##1}}}
\expandafter\def\csname PY@tok@vi\endcsname{\def\PY@tc##1{\textcolor[rgb]{0.10,0.09,0.49}{##1}}}
\expandafter\def\csname PY@tok@vm\endcsname{\def\PY@tc##1{\textcolor[rgb]{0.10,0.09,0.49}{##1}}}
\expandafter\def\csname PY@tok@sa\endcsname{\def\PY@tc##1{\textcolor[rgb]{0.73,0.13,0.13}{##1}}}
\expandafter\def\csname PY@tok@sb\endcsname{\def\PY@tc##1{\textcolor[rgb]{0.73,0.13,0.13}{##1}}}
\expandafter\def\csname PY@tok@sc\endcsname{\def\PY@tc##1{\textcolor[rgb]{0.73,0.13,0.13}{##1}}}
\expandafter\def\csname PY@tok@dl\endcsname{\def\PY@tc##1{\textcolor[rgb]{0.73,0.13,0.13}{##1}}}
\expandafter\def\csname PY@tok@s2\endcsname{\def\PY@tc##1{\textcolor[rgb]{0.73,0.13,0.13}{##1}}}
\expandafter\def\csname PY@tok@sh\endcsname{\def\PY@tc##1{\textcolor[rgb]{0.73,0.13,0.13}{##1}}}
\expandafter\def\csname PY@tok@s1\endcsname{\def\PY@tc##1{\textcolor[rgb]{0.73,0.13,0.13}{##1}}}
\expandafter\def\csname PY@tok@mb\endcsname{\def\PY@tc##1{\textcolor[rgb]{0.40,0.40,0.40}{##1}}}
\expandafter\def\csname PY@tok@mf\endcsname{\def\PY@tc##1{\textcolor[rgb]{0.40,0.40,0.40}{##1}}}
\expandafter\def\csname PY@tok@mh\endcsname{\def\PY@tc##1{\textcolor[rgb]{0.40,0.40,0.40}{##1}}}
\expandafter\def\csname PY@tok@mi\endcsname{\def\PY@tc##1{\textcolor[rgb]{0.40,0.40,0.40}{##1}}}
\expandafter\def\csname PY@tok@il\endcsname{\def\PY@tc##1{\textcolor[rgb]{0.40,0.40,0.40}{##1}}}
\expandafter\def\csname PY@tok@mo\endcsname{\def\PY@tc##1{\textcolor[rgb]{0.40,0.40,0.40}{##1}}}
\expandafter\def\csname PY@tok@ch\endcsname{\let\PY@it=\textit\def\PY@tc##1{\textcolor[rgb]{0.25,0.50,0.50}{##1}}}
\expandafter\def\csname PY@tok@cm\endcsname{\let\PY@it=\textit\def\PY@tc##1{\textcolor[rgb]{0.25,0.50,0.50}{##1}}}
\expandafter\def\csname PY@tok@cpf\endcsname{\let\PY@it=\textit\def\PY@tc##1{\textcolor[rgb]{0.25,0.50,0.50}{##1}}}
\expandafter\def\csname PY@tok@c1\endcsname{\let\PY@it=\textit\def\PY@tc##1{\textcolor[rgb]{0.25,0.50,0.50}{##1}}}
\expandafter\def\csname PY@tok@cs\endcsname{\let\PY@it=\textit\def\PY@tc##1{\textcolor[rgb]{0.25,0.50,0.50}{##1}}}

\def\PYZbs{\char`\\}
\def\PYZus{\char`\_}
\def\PYZob{\char`\{}
\def\PYZcb{\char`\}}
\def\PYZca{\char`\^}
\def\PYZam{\char`\&}
\def\PYZlt{\char`\<}
\def\PYZgt{\char`\>}
\def\PYZsh{\char`\#}
\def\PYZpc{\char`\%}
\def\PYZdl{\char`\$}
\def\PYZhy{\char`\-}
\def\PYZsq{\char`\'}
\def\PYZdq{\char`\"}
\def\PYZti{\char`\~}
% for compatibility with earlier versions
\def\PYZat{@}
\def\PYZlb{[}
\def\PYZrb{]}
\makeatother


    % Exact colors from NB
    \definecolor{incolor}{rgb}{0.0, 0.0, 0.5}
    \definecolor{outcolor}{rgb}{0.545, 0.0, 0.0}



    
    % Prevent overflowing lines due to hard-to-break entities
    \sloppy 
    % Setup hyperref package
    \hypersetup{
      breaklinks=true,  % so long urls are correctly broken across lines
      colorlinks=true,
      urlcolor=urlcolor,
      linkcolor=linkcolor,
      citecolor=citecolor,
      }
    % Slightly bigger margins than the latex defaults
    
    \geometry{verbose,tmargin=1in,bmargin=1in,lmargin=1in,rmargin=1in}
    
    

    \begin{document}
    
    
    \maketitle
    
    

    
    \section{Basic Probability Review}\label{basic-probability-review}

\textbf{Ref}

{[}1{]} https://github.com/AM207/2016

\begin{center}\rule{0.5\linewidth}{\linethickness}\end{center}

    \begin{Verbatim}[commandchars=\\\{\}]
{\color{incolor}In [{\color{incolor}1}]:} \PY{c+c1}{\PYZsh{}\PYZsh{} do all neccessary imports}
        \PY{k+kn}{import} \PY{n+nn}{numpy} \PY{k}{as} \PY{n+nn}{np}
        \PY{k+kn}{import} \PY{n+nn}{matplotlib}
        \PY{o}{\PYZpc{}}\PY{k}{matplotlib} inline
        \PY{k+kn}{import} \PY{n+nn}{matplotlib}\PY{n+nn}{.}\PY{n+nn}{pyplot} \PY{k}{as} \PY{n+nn}{plt}
        \PY{k+kn}{import} \PY{n+nn}{seaborn} \PY{k}{as} \PY{n+nn}{sns}
        \PY{n}{sns}\PY{o}{.}\PY{n}{set\PYZus{}style}\PY{p}{(}\PY{l+s+s2}{\PYZdq{}}\PY{l+s+s2}{white}\PY{l+s+s2}{\PYZdq{}}\PY{p}{)}
        \PY{n}{sns}\PY{o}{.}\PY{n}{set\PYZus{}context}\PY{p}{(}\PY{l+s+s2}{\PYZdq{}}\PY{l+s+s2}{notebook}\PY{l+s+s2}{\PYZdq{}}\PY{p}{)}
        \PY{k+kn}{import} \PY{n+nn}{pandas} \PY{k}{as} \PY{n+nn}{pd}
        \PY{k+kn}{from}  \PY{n+nn}{scipy} \PY{k}{import} \PY{n}{stats} 
        
        \PY{k+kn}{import} \PY{n+nn}{math}
\end{Verbatim}


    \section{Short review of probability theory, distributions
etc}\label{short-review-of-probability-theory-distributions-etc}

Probabilities are numbers that tell us how often things happen
(frequentist) or our believe in a different outcome or notion
(Bayesian). These two views of probabilities (and which one is
preferable) are a very disputed topic. In this course we take a very
simple approach and we use whatever is convenient at the time, or
whatever is better to demonstrate the methodologies we are learning.

There are a few basic rules that form the basis for all probabilistic
methodology. In this lecture, we just briefly review these rules. For an
in depth introduction into probability theory, you can for example
access the \href{http://projects.iq.harvard.edu/stat110/youtube}{stat110
material online}, read Joe Blitzstein's
\href{http://www.amazon.com/gp/product/1466575573/ref=as_li_tl?ie=UTF8\&camp=1789\&creative=390957\&creativeASIN=1466575573\&linkCode=as2}{Introduction
to Probability}, or also Larrry Wasserman's
\href{http://www.amazon.com/All-Statistics-Statistical-Inference-Springer/dp/0387402721}{All
of Statistics}.

Let us define some terminology: \(X\) and \(Y\) are two events and
\(p(X)\) is the probability of the event \(X\) to happen. \$X\^{}c \$ is
the complement of \(X\), the event which is all the occurrences which
are not in \(X\). \(X \cup Y\) is the union of \(X\) and \(Y\);
\(X \cap Y\) is the intersection of \(X\) and \(Y\). (Both \(X \cup Y\)
and \(X\cap Y\) are also events.)

\textbf{ex} \[X = \{ \hbox{ students who has his/her cell phone} \}\]

\subsubsection{The very fundamental rules of
probability:}\label{the-very-fundamental-rules-of-probability}

\begin{enumerate}
\def\labelenumi{\arabic{enumi}.}
\item
  \(p(X) = 1 \;\) \(X\) has to happen almost surely
\item
  \(p(X) = 0 \;\) \(X\) will certainly not happen almost surely
\item
  \(0 ≤ p(X) ≤ 1 \;\) \(X\) has probability range from low to high
\item
  \(p(X)+p(X^c)=1 \;\) \(X\) must either happen or not happen
\item
  \(p(X \cup Y)=p(X)+p(Y)−p(X \cap Y) \;\) \(X\) can happen and \(Y\)
  can happen but we must subtract the cases that are happening together
  so we do not over-count.
\end{enumerate}

    For two random variables \(x,y\) the \(p(x,y)\) is called the joint
distribution, and \(p(x|y)\) the conditional distribution.

\subsubsection{Sum rule (marginal
distribution)}\label{sum-rule-marginal-distribution}

We can write the marginal probability of \(x\) as a sum over the joint
distribution of \(x\) and \(y\) where we sum over all possibilities of
\(y\),

\[p(x) = \sum_y p(x,y) \]

for continuous random variables this becomes:

\[ p(x) = \int_y p(x,y) \, dy \]

    \subsubsection{Product rule}\label{product-rule}

We can rewrite a joint distribution as a product of a conditional and
marginal probability,

\[ p(x,y) = p(x|y) p(y) \]

    \subsubsection{Chain rule}\label{chain-rule}

The product rule is applied repeatedly to give expressions for the joint
probability involving more than two variables. For example, the joint
distribution over three variables can be factorized into a product of
conditional probabilities:

\[ p(x,y,z) = p(x|y,z) \, p(y,z) = p(x |y,z) \, p(y|z) p(z) \]

    \subsubsection{Bayes rule}\label{bayes-rule}

Given the product rule one can derive the Bayes rule, which plays a
central role in a lot of the things we will be talking about:

\[ p(y|x) = \frac{p(x|y) \, p(y) }{p(x)} = \frac{p(x|y) \, p(y) }{\sum_{y'} p(x,y')} = \frac{p(x|y) \, p(y) }{\sum_{y'} p(x|y')p(y')}\]

    \subsubsection{Independence}\label{independence}

Two variables are said to be independent if their joint distribution
factorizes into a product of two marginal probabilities:

\[ p(x,y) = p(x) \, p(y) \]

Note that if two variables are uncorrelated, that does not mean they are
statistically independent. There are many ways to measure statistical
association between variables and correlation is just one of them.
However, if two variables are independent, this will ensure there is no
correlation between them. Another consequence of independence is that if
\(x\) and \(y\) are independent, the conditional probability of \(x\)
given \(y\) is just the probability of \(x\):

\[ p(x|y) = p(x) \]

In other words, by conditioning on a particular \(y\), we have learned
nothing about \(x\) because of independence. Two variables \(x\) and
\(y\) and said to be conditionally independent of \(z\) if the following
holds:

\[ p(x,y|z) = p(x|z) p(y|z) \]

Therefore, if we learn about z, x and y become independent. Another way
to write that \(x\) and \(y\) are conditionally independent of \(z\) is

\[ p(x| z, y) = p(x|z) \]

In other words, if we condition on \(z\), and now also learn about
\(y\), this is not going to change the probability of \(x\). It is
important to realize that conditional independence between \(x\) and
\(y\) does not imply independence between \(x\) and \(y\).

    \subsection{Distributions}\label{distributions}

A probability distribution (aka probability measure) is a function that
takes an event and gives its probability. Sometimes this is not well
defined in the whole probability space but we will not worry about this
for now. There are two classes of statistical distributions, discete and
continous.

    \subsubsection{Discrete distributions}\label{discrete-distributions}

    \subsubsection{Bernoulli distribution}\label{bernoulli-distribution}

The probability for a yes/no outcome of an experiment, is given by the
Bernoulli distribution. The Bernoulli distribution is the mother of all
distributions. Every experiment, can always be expressed in terms of
success/failure. If you do not know which distribution to use, you can
think of any problem as a yes/no problem and starting from there you
work your way to all other distributions.

Let \(k\) be the outcome of our experiment. Then \(p\) is the
probability of a success (\(k=1\)), which means we expect the value
\(k=1\), \(p n\) times out of \(n\). The probability for a failure
(\(k=0\)) is \(1-p\). For example if we have \(p=0.6\) and we repeat our
experiment 10 times, we expect 6 experiments to be successful and 4 to
be not successful.

\[ f(k;p) = \begin{cases} p & \text{if }k=1, \\[6pt]
1-p & \text {if }k=0.\end{cases} \]

This can also be expressed as:

\[ f(k;p) = p^k (1-p)^{1-k}\!\quad \text{for }k\in\{0,1\}.\]

The expected value of a Bernoulli random variable \(X\) is \(E[X] = p\)
and the variance is \(Var[X]=p(1-p)\)

To see this, remember that the expectated value is calculated by
\(E[X] = \sum_k k \cdot p(k) = 1 \cdot p + 0 \cdot (1-p) = p\) and the
variance is \$\mathrm{var}(X) = E{[}X\^{}2{]} - E{[}X{]}\^{}2 = p -
p\^{}2 = p(1-p) \$

    \paragraph{How to generate some random numbers from a Bernoulli
distribution}\label{how-to-generate-some-random-numbers-from-a-bernoulli-distribution}

    We use
\href{http://docs.scipy.org/doc/scipy-0.14.0/reference/stats.html}{scipy.stats}
as our library to generate random numbers. For the Bernoulli
distribution the library provides the function \texttt{.rvs(p,\ size=1)}
to generate random numbers (Random Variates), and \texttt{.pmf(x,p)} to
compute the Probability Mass Function.

    \begin{Verbatim}[commandchars=\\\{\}]
{\color{incolor}In [{\color{incolor}2}]:} \PY{c+c1}{\PYZsh{}\PYZsh{} Draw and print 10 experiment outcomes for p=0.6}
        \PY{n+nb}{print}\PY{p}{(}\PY{n}{stats}\PY{o}{.}\PY{n}{bernoulli}\PY{o}{.}\PY{n}{rvs}\PY{p}{(}\PY{n}{p}\PY{o}{=}\PY{l+m+mf}{0.6}\PY{p}{,} \PY{n}{size}\PY{o}{=}\PY{l+m+mi}{10}\PY{p}{)}\PY{p}{)}
        
        \PY{c+c1}{\PYZsh{}\PYZsh{} compute the probability mass function for p=0.7}
        \PY{n}{stats}\PY{o}{.}\PY{n}{bernoulli}\PY{o}{.}\PY{n}{pmf}\PY{p}{(}\PY{p}{[}\PY{l+m+mi}{0}\PY{p}{,}\PY{l+m+mi}{1}\PY{p}{]}\PY{p}{,} \PY{n}{p}\PY{o}{=}\PY{l+m+mf}{0.7}\PY{p}{)}
\end{Verbatim}


    \begin{Verbatim}[commandchars=\\\{\}]
[1 1 1 1 1 1 1 0 0 1]

    \end{Verbatim}

\begin{Verbatim}[commandchars=\\\{\}]
{\color{outcolor}Out[{\color{outcolor}2}]:} array([0.3, 0.7])
\end{Verbatim}
            
    \paragraph{Example Bernoulli
Distribution}\label{example-bernoulli-distribution}

Suppose in last year the class got C or better with probability 0.75.
Let the random variable X be the grade with the probability that someone
gets a grade higher than C.

    \begin{Verbatim}[commandchars=\\\{\}]
{\color{incolor}In [{\color{incolor}3}]:} \PY{c+c1}{\PYZsh{} get the values}
        \PY{n}{grades} \PY{o}{=} \PY{n}{stats}\PY{o}{.}\PY{n}{bernoulli}\PY{o}{.}\PY{n}{rvs}\PY{p}{(}\PY{l+m+mf}{0.75}\PY{p}{,} \PY{n}{size}\PY{o}{=}\PY{l+m+mi}{100}\PY{p}{)}
        \PY{n+nb}{print}\PY{p}{(}\PY{n}{grades}\PY{p}{)}
        \PY{n}{binValues}\PY{p}{,} \PY{n}{binBorders} \PY{o}{=} \PY{n}{np}\PY{o}{.}\PY{n}{histogram}\PY{p}{(}\PY{n}{grades}\PY{p}{,} \PY{n}{bins}\PY{o}{=}\PY{l+m+mi}{2}\PY{p}{)}
        \PY{n+nb}{print}\PY{p}{(}\PY{n}{binValues}\PY{p}{,} \PY{n}{binBorders}\PY{p}{)}
        
        
        \PY{c+c1}{\PYZsh{} plot the result}
        \PY{n}{plt}\PY{o}{.}\PY{n}{bar}\PY{p}{(}\PY{p}{[}\PY{l+m+mi}{0}\PY{p}{,}\PY{l+m+mi}{1}\PY{p}{]}\PY{p}{,}\PY{n}{binValues}\PY{p}{,} \PY{n}{width}\PY{o}{=}\PY{l+m+mf}{0.5}\PY{p}{,} \PY{n}{align}\PY{o}{=}\PY{l+s+s2}{\PYZdq{}}\PY{l+s+s2}{center}\PY{l+s+s2}{\PYZdq{}}\PY{p}{)}
        \PY{n}{plt}\PY{o}{.}\PY{n}{xticks}\PY{p}{(}\PY{p}{[}\PY{l+m+mi}{0}\PY{p}{,}\PY{l+m+mi}{1}\PY{p}{]}\PY{p}{,} \PY{p}{(}\PY{l+s+s1}{\PYZsq{}}\PY{l+s+s1}{Fail}\PY{l+s+s1}{\PYZsq{}}\PY{p}{,}\PY{l+s+s1}{\PYZsq{}}\PY{l+s+s1}{Pass}\PY{l+s+s1}{\PYZsq{}}\PY{p}{)} \PY{p}{)}
        
        \PY{n+nb}{print}\PY{p}{(}\PY{l+s+s2}{\PYZdq{}}\PY{l+s+s2}{Number of samples in each bin: }\PY{l+s+s2}{\PYZdq{}}\PY{p}{,} \PY{n}{binValues}\PY{p}{)}
\end{Verbatim}


    \begin{Verbatim}[commandchars=\\\{\}]
[1 1 1 1 1 0 1 1 1 1 1 0 0 1 0 1 1 0 1 1 1 1 1 1 1 1 0 1 1 1 1 0 1 1 0 1 0
 1 1 1 0 1 1 1 1 1 1 1 1 0 1 1 1 1 1 0 1 1 0 1 1 0 0 0 1 1 1 0 1 1 1 1 1 1
 1 0 0 0 0 0 1 0 1 1 1 1 1 0 1 0 1 1 1 1 1 1 1 1 1 0]
[26 74] [0.  0.5 1. ]
Number of samples in each bin:  [26 74]

    \end{Verbatim}

    \begin{center}
    \adjustimage{max size={0.9\linewidth}{0.9\paperheight}}{output_15_1.png}
    \end{center}
    { \hspace*{\fill} \\}
    
    \subsection{Binomial distribution}\label{binomial-distribution}

While the Bernoulli distribution describes the outcome of a single
experiment, the Binomial distribution is the distribution of the number
of successes in a sequence of \(n\) independent yes/no experiments, each
of which yields success with probability \(p\). Thus, the Bernoulli
distribution is equal to the Binomial distribution when \(n=1\).

\[f(k; n, p) = {n\choose k}p^k(1-p)^{n-k} \]

where

\[{n\choose k}=\frac{n!}{k!(n-k)!}\]

The expected value of a random variable \(X\) is \(E[X]=np\) and the
variance is \(Var[X]=np(1-p)\)

    The following plot shows histograms for data generated from a Binomial
distribution with different values for \(p\).

    \begin{Verbatim}[commandchars=\\\{\}]
{\color{incolor}In [{\color{incolor}4}]:} \PY{c+c1}{\PYZsh{} generate and plot random variates for different values of p}
        \PY{k}{for} \PY{n}{p} \PY{o+ow}{in} \PY{p}{[}\PY{l+m+mf}{0.1}\PY{p}{,} \PY{l+m+mf}{0.3}\PY{p}{,} \PY{l+m+mf}{0.6}\PY{p}{,} \PY{l+m+mf}{0.8}\PY{p}{]}\PY{p}{:}
            \PY{n}{randomVariates} \PY{o}{=} \PY{n}{stats}\PY{o}{.}\PY{n}{binom}\PY{o}{.}\PY{n}{rvs}\PY{p}{(}\PY{l+m+mi}{100}\PY{p}{,} \PY{n}{p}\PY{p}{,} \PY{n}{size}\PY{o}{=}\PY{l+m+mi}{1000}\PY{p}{)}
            \PY{n}{plt}\PY{o}{.}\PY{n}{hist}\PY{p}{(}\PY{n}{randomVariates}\PY{p}{,} \PY{n}{alpha}\PY{o}{=}\PY{l+m+mf}{0.5}\PY{p}{,} \PY{n}{label}\PY{o}{=}\PY{l+s+s1}{\PYZsq{}}\PY{l+s+s1}{p=}\PY{l+s+s1}{\PYZsq{}} \PY{o}{+} \PY{n}{np}\PY{o}{.}\PY{n}{str}\PY{p}{(}\PY{n}{p}\PY{p}{)}\PY{p}{)}
        
        \PY{n}{plt}\PY{o}{.}\PY{n}{legend}\PY{p}{(}\PY{p}{)}
        \PY{c+c1}{\PYZsh{}plt.show()}
\end{Verbatim}


\begin{Verbatim}[commandchars=\\\{\}]
{\color{outcolor}Out[{\color{outcolor}4}]:} <matplotlib.legend.Legend at 0x11ad58eb8>
\end{Verbatim}
            
    \begin{center}
    \adjustimage{max size={0.9\linewidth}{0.9\paperheight}}{output_18_1.png}
    \end{center}
    { \hspace*{\fill} \\}
    
    \paragraph{Example Binomial
distribution}\label{example-binomial-distribution}

What is the probability of 30 people getting a C or higher this year
assuming we have 50 students in total?

    \begin{Verbatim}[commandchars=\\\{\}]
{\color{incolor}In [{\color{incolor}5}]:} \PY{n}{numberOfStudents} \PY{o}{=} \PY{l+m+mi}{30}
        \PY{n}{totalNumberOfStudents} \PY{o}{=} \PY{l+m+mi}{50}
        \PY{n}{p} \PY{o}{=} \PY{l+m+mf}{0.75}
        
        \PY{n+nb}{print}\PY{p}{(}\PY{l+s+s2}{\PYZdq{}}\PY{l+s+s2}{The probability of 30 students passing course is:}\PY{l+s+s2}{\PYZdq{}}\PY{p}{)}
        \PY{n+nb}{print}\PY{p}{(}\PY{n}{stats}\PY{o}{.}\PY{n}{binom}\PY{o}{.}\PY{n}{pmf}\PY{p}{(}\PY{n}{numberOfStudents}\PY{p}{,} \PY{n}{totalNumberOfStudents}\PY{p}{,} \PY{n}{p}\PY{p}{)}\PY{p}{)}
\end{Verbatim}


    \begin{Verbatim}[commandchars=\\\{\}]
The probability of 30 students passing course is:
0.0076547014310495804

    \end{Verbatim}

    That seems rather low, if we forget that we asked for the probability of
exactly 8 students passing course, not at least 8 students. Let's look
at the full histogram.

    \begin{Verbatim}[commandchars=\\\{\}]
{\color{incolor}In [{\color{incolor}6}]:} \PY{n}{randomVariates} \PY{o}{=} \PY{n}{stats}\PY{o}{.}\PY{n}{binom}\PY{o}{.}\PY{n}{rvs}\PY{p}{(}\PY{n}{totalNumberOfStudents}\PY{p}{,} \PY{n}{p}\PY{o}{=}\PY{n}{p}\PY{p}{,} \PY{n}{size}\PY{o}{=}\PY{l+m+mi}{10000}\PY{p}{)}
        \PY{n}{plt}\PY{o}{.}\PY{n}{hist}\PY{p}{(}\PY{n}{randomVariates}\PY{p}{,} \PY{n}{bins}\PY{o}{=}\PY{n+nb}{range}\PY{p}{(}\PY{l+m+mi}{0}\PY{p}{,}\PY{n}{totalNumberOfStudents}\PY{o}{+}\PY{l+m+mi}{1}\PY{p}{)}\PY{p}{,} \PY{n}{alpha}\PY{o}{=}\PY{l+m+mf}{0.5}\PY{p}{,} \PY{n}{label}\PY{o}{=}\PY{l+s+s1}{\PYZsq{}}\PY{l+s+s1}{p=}\PY{l+s+s1}{\PYZsq{}} \PY{o}{+} \PY{n}{np}\PY{o}{.}\PY{n}{str}\PY{p}{(}\PY{n}{p}\PY{p}{)}\PY{p}{)}
        
        \PY{n}{plt}\PY{o}{.}\PY{n}{legend}\PY{p}{(}\PY{p}{)}
\end{Verbatim}


\begin{Verbatim}[commandchars=\\\{\}]
{\color{outcolor}Out[{\color{outcolor}6}]:} <matplotlib.legend.Legend at 0x11adc6630>
\end{Verbatim}
            
    \begin{center}
    \adjustimage{max size={0.9\linewidth}{0.9\paperheight}}{output_22_1.png}
    \end{center}
    { \hspace*{\fill} \\}
    
    As you can see the probability of 30 students passing is indeed rather
low, but the probability of 40 students passing is much higher. What if
we would like to know the probability of at least 30 students passing?
We could look at the cumulative distribution and go from there, or we
can use our samples from above and just count.

    \begin{Verbatim}[commandchars=\\\{\}]
{\color{incolor}In [{\color{incolor}7}]:} \PY{n+nb}{print}\PY{p}{(}\PY{l+s+s2}{\PYZdq{}}\PY{l+s+s2}{Probability of at least 30 students passing: }\PY{l+s+s2}{\PYZdq{}}\PY{p}{,} \PY{n}{np}\PY{o}{.}\PY{n}{sum}\PY{p}{(}\PY{n}{randomVariates}\PY{o}{\PYZgt{}}\PY{n}{numberOfStudents}\PY{p}{)}\PY{o}{/}\PY{n+nb}{float}\PY{p}{(}\PY{n+nb}{len}\PY{p}{(}\PY{n}{randomVariates}\PY{p}{)}\PY{p}{)}\PY{p}{)}
\end{Verbatim}


    \begin{Verbatim}[commandchars=\\\{\}]
Probability of at least 30 students passing:  0.9841

    \end{Verbatim}

    \subsection{Poisson distribution}\label{poisson-distribution}

The Poisson distribution is another discrete distribution, it expresses
the probability of a given number of events occurring in a fixed
interval of time (or space, volume, etc.). One assumption made is that
these events occur with a known average rate and independently of each
other. An example is the number of electrons detected by a sensor in an
electron microscope during a time interval, or the number of soldiers in
the Prussian army killed accidentally by horse kicks
\href{http://en.wikipedia.org/wiki/Poisson_distribution}{(see here)}.

The Poisson distribution is defined as:

\[ f(k; \lambda)= \frac{\lambda^k e^{-\lambda}}{k!}, \]

where \(k\) is the number of events, \(\lambda\) is a positive real
number, and \(e\) is Euler's number (\(e = 2.71828 \ldots\)).

    \begin{Verbatim}[commandchars=\\\{\}]
{\color{incolor}In [{\color{incolor}8}]:} \PY{c+c1}{\PYZsh{} generate samples for different values of lambda}
        \PY{k}{for} \PY{n}{lambdaParameter} \PY{o+ow}{in} \PY{p}{[}\PY{l+m+mi}{1}\PY{p}{,}\PY{l+m+mi}{6}\PY{p}{,}\PY{l+m+mi}{14}\PY{p}{]}\PY{p}{:}
            \PY{n}{randomVariates} \PY{o}{=} \PY{n}{stats}\PY{o}{.}\PY{n}{poisson}\PY{o}{.}\PY{n}{rvs}\PY{p}{(}\PY{n}{lambdaParameter}\PY{p}{,} \PY{n}{size}\PY{o}{=}\PY{l+m+mi}{1000}\PY{p}{)}
            \PY{n}{plt}\PY{o}{.}\PY{n}{hist}\PY{p}{(}\PY{n}{randomVariates}\PY{p}{,} \PY{n}{alpha}\PY{o}{=}\PY{l+m+mf}{0.5}\PY{p}{,} \PY{n}{bins}\PY{o}{=}\PY{n+nb}{range}\PY{p}{(}\PY{l+m+mi}{0}\PY{p}{,}\PY{l+m+mi}{26}\PY{p}{)}\PY{p}{,} \PY{n}{label}\PY{o}{=}\PY{l+s+s1}{\PYZsq{}}\PY{l+s+s1}{\PYZdl{}}\PY{l+s+s1}{\PYZbs{}}\PY{l+s+s1}{lambda=}\PY{l+s+s1}{\PYZsq{}} \PY{o}{+} \PY{n}{np}\PY{o}{.}\PY{n}{str}\PY{p}{(}\PY{n}{lambdaParameter}\PY{p}{)} \PY{o}{+} \PY{l+s+s1}{\PYZsq{}}\PY{l+s+s1}{\PYZdl{}}\PY{l+s+s1}{\PYZsq{}}\PY{p}{)}
        
        \PY{n}{plt}\PY{o}{.}\PY{n}{legend}\PY{p}{(}\PY{p}{)}
\end{Verbatim}


\begin{Verbatim}[commandchars=\\\{\}]
{\color{outcolor}Out[{\color{outcolor}8}]:} <matplotlib.legend.Legend at 0x11af05438>
\end{Verbatim}
            
    \begin{center}
    \adjustimage{max size={0.9\linewidth}{0.9\paperheight}}{output_26_1.png}
    \end{center}
    { \hspace*{\fill} \\}
    
    \subsection{Continuous variables}\label{continuous-variables}

The difficulty with continuous random variables is that you can't find
the probability of the exact event. The problem is that every value of a
concrete random variate is \(0\), but if you add enough values together
you have something with a positive value. So the usual way around this
difficulty, is to define probabilities for intervals.

    \subsubsection{Normal Distribution}\label{normal-distribution}

Probably the most important of all distributions is the normal
distribution. Many phenomena in nature follow this distribution. Due to
the Central Limit Theorem which states that, the distribution of a sum
of random variables can be approximated by a normal distribution. A
Normal distribution with mean \(\mu\) and standard deviation \(\sigma\)
is defined as:

\[ f(x; \mu, \sigma) = \frac{1}{\sigma\sqrt{2\pi}} e^{ -\frac{(x-\mu)^2}{2\sigma^2} }, \]

So if your data is the sum of many independent processes it is most
likely normally distributed (or very close to it). An important example
are measurement errors. Let's look at some normal distributions.

    \begin{Verbatim}[commandchars=\\\{\}]
{\color{incolor}In [{\color{incolor}9}]:} \PY{n}{x} \PY{o}{=} \PY{n}{np}\PY{o}{.}\PY{n}{linspace}\PY{p}{(}\PY{o}{\PYZhy{}}\PY{l+m+mi}{10}\PY{p}{,}\PY{l+m+mi}{20}\PY{p}{,} \PY{n}{num}\PY{o}{=}\PY{l+m+mi}{200}\PY{p}{)}
        
        \PY{k}{for} \PY{n}{mu}\PY{p}{,} \PY{n}{sigma} \PY{o+ow}{in} \PY{n+nb}{zip}\PY{p}{(}\PY{p}{[}\PY{o}{\PYZhy{}}\PY{l+m+mi}{3}\PY{p}{,} \PY{l+m+mi}{3}\PY{p}{,} \PY{l+m+mi}{10}\PY{p}{]}\PY{p}{,} \PY{p}{[}\PY{l+m+mi}{1}\PY{p}{,} \PY{l+m+mi}{2}\PY{p}{,} \PY{l+m+mi}{5}\PY{p}{]}\PY{p}{)}\PY{p}{:}
            \PY{n}{plt}\PY{o}{.}\PY{n}{plot}\PY{p}{(}\PY{n}{x}\PY{p}{,} \PY{n}{stats}\PY{o}{.}\PY{n}{norm}\PY{o}{.}\PY{n}{pdf}\PY{p}{(}\PY{n}{x}\PY{p}{,} \PY{n}{mu}\PY{p}{,} \PY{n}{sigma}\PY{p}{)}\PY{p}{,} \PY{n}{lw}\PY{o}{=}\PY{l+m+mi}{2}\PY{p}{,} 
                     \PY{n}{label} \PY{o}{=} \PY{l+s+sa}{r}\PY{l+s+s2}{\PYZdq{}}\PY{l+s+s2}{\PYZdl{}}\PY{l+s+s2}{\PYZbs{}}\PY{l+s+s2}{mu = }\PY{l+s+si}{\PYZob{}0:.1f\PYZcb{}}\PY{l+s+s2}{, }\PY{l+s+s2}{\PYZbs{}}\PY{l+s+s2}{sigma=}\PY{l+s+si}{\PYZob{}1:.1f\PYZcb{}}\PY{l+s+s2}{\PYZdl{}}\PY{l+s+s2}{\PYZdq{}}\PY{o}{.}\PY{n}{format}\PY{p}{(}\PY{n}{mu}\PY{p}{,} \PY{n}{sigma}\PY{p}{)}\PY{p}{)}
        \PY{n}{plt}\PY{o}{.}\PY{n}{legend}\PY{p}{(}\PY{p}{)}
\end{Verbatim}


\begin{Verbatim}[commandchars=\\\{\}]
{\color{outcolor}Out[{\color{outcolor}9}]:} <matplotlib.legend.Legend at 0x11b031828>
\end{Verbatim}
            
    \begin{center}
    \adjustimage{max size={0.9\linewidth}{0.9\paperheight}}{output_29_1.png}
    \end{center}
    { \hspace*{\fill} \\}
    
    \subsubsection{Exponential Distribution}\label{exponential-distribution}

The exponential distribution is the probability distribution that
describes the time between events in a Poisson process,i.e. a process in
which events occur continuously and independently at a constant average
rate.

Note that the exponential distribution is not the same as the class of
exponential families of distributions, which is a large class of
probability distributions that includes the exponential distribution as
one of its members, but also includes the normal distribution, binomial
distribution, gamma distribution, Poisson, and many others.

\[f(x;\lambda) = \begin{cases}
\lambda e^{-\lambda x} & x \ge 0, \\
0 & x < 0.
\end{cases}\]

Therefore, the random variable X has an exponential distribution with
parameter \(\lambda\), we say \(X\) is exponential and write \$
X∼Exp(\lambda)\$ Given a specific \(\lambda\), the expected value of an
exponential random variable is equal to the inverse of \(\lambda\), that
is: \(E[X|\lambda]=\frac{1}{\lambda}\)

    \begin{Verbatim}[commandchars=\\\{\}]
{\color{incolor}In [{\color{incolor}10}]:} \PY{n}{x} \PY{o}{=} \PY{n}{np}\PY{o}{.}\PY{n}{linspace}\PY{p}{(}\PY{l+m+mi}{0}\PY{p}{,}\PY{l+m+mi}{4}\PY{p}{,} \PY{l+m+mi}{100}\PY{p}{)}
         \PY{n}{lambdaParameter} \PY{o}{=} \PY{p}{[}\PY{l+m+mf}{0.5}\PY{p}{,} \PY{l+m+mi}{1}\PY{p}{,} \PY{l+m+mi}{2}\PY{p}{,} \PY{l+m+mi}{4}\PY{p}{]}
         \PY{k}{for} \PY{n}{l} \PY{o+ow}{in} \PY{n}{lambdaParameter}\PY{p}{:}
             \PY{n}{plt}\PY{o}{.}\PY{n}{plot}\PY{p}{(}\PY{n}{x}\PY{p}{,} \PY{n}{stats}\PY{o}{.}\PY{n}{expon}\PY{o}{.}\PY{n}{pdf}\PY{p}{(}\PY{n}{x}\PY{p}{,} \PY{n}{scale}\PY{o}{=}\PY{l+m+mf}{1.}\PY{o}{/}\PY{n}{l}\PY{p}{)}\PY{p}{,} \PY{n}{lw}\PY{o}{=}\PY{l+m+mi}{2}\PY{p}{,} \PY{n}{label} \PY{o}{=} \PY{l+s+s2}{\PYZdq{}}\PY{l+s+s2}{\PYZdl{}}\PY{l+s+s2}{\PYZbs{}}\PY{l+s+s2}{lambda = }\PY{l+s+si}{\PYZpc{}.1f}\PY{l+s+s2}{\PYZdl{}}\PY{l+s+s2}{\PYZdq{}}\PY{o}{\PYZpc{}}\PY{k}{l})    
         \PY{n}{plt}\PY{o}{.}\PY{n}{legend}\PY{p}{(}\PY{p}{)}
\end{Verbatim}


\begin{Verbatim}[commandchars=\\\{\}]
{\color{outcolor}Out[{\color{outcolor}10}]:} <matplotlib.legend.Legend at 0x11c2725f8>
\end{Verbatim}
            
    \begin{center}
    \adjustimage{max size={0.9\linewidth}{0.9\paperheight}}{output_31_1.png}
    \end{center}
    { \hspace*{\fill} \\}
    

    % Add a bibliography block to the postdoc
    
    
    
    \end{document}
